\documentclass[a4paper,12pt]{article}
 \usepackage{amscd,amsmath,amssymb,amsthm,enumerate,indentfirst,longtable}
\usepackage{mathspec}
\usepackage{xunicode, xltxtra}
%\usepackage[russian]{babel}
\usepackage{polyglossia}
\setdefaultlanguage{russian}
\setotherlanguage{english}
\usepackage{longtable}
\usepackage{mathptmx}
\usepackage{authblk}
\usepackage{caption}
\captionsetup{font=large,justification=centerlast,labelsep=period}
\usepackage[unicode]{hyperref}
\usepackage{datetime}
\usepackage{graphics}
\usepackage{afterpage}
\usepackage{tabularx}
\usepackage{lastpage}
\usepackage{totcount}
\usepackage{algorithm}
%\usepackage{algorithmicx}
\usepackage{algpseudocode}

\newcommand{\No}{\textnumero}


\newcommand{\ZZ}{\mathbb{Z}}
\newcommand{\NN}{\mathbb{N}}
\newcommand{\RR}{\mathbb{R}}



\DeclareMathOperator{\ord}{\mathrm{ord}}



\DeclareMathOperator{\aitilda}{\widetilde{AI}} %
\DeclareMathOperator{\ai}{AI} %
\DeclareMathOperator{\ann}{Ann} %
\DeclareMathOperator{\dist}{dist} %
\DeclareMathOperator{\imp}{imp} %
\DeclareMathOperator{\nl}{nl} %
\DeclareMathOperator{\tann}{TAnn} %
\DeclareMathOperator{\wt}{wt} %
\DeclareMathOperator{\sr}{SR} %
\DeclareMathOperator{\lfsr}{LFSR} %
\DeclareMathOperator{\rang}{rang} %
\DeclareMathOperator{\prop}{Prop} %
\DeclareMathOperator{\Char}{char}
\newcommand{\ftwo}{\mathbb{F}_2} %
\newcommand{\fntwo}{\mathbb{F}_2^n} %
\newcommand{\fttwo}{\mathbb{F}_2^t} %
\newcommand{\srf}{\sr(f)} %
\newcommand{\lfsrf}{\lfsr(\chi,f)} %
\newcommand{\gotu}{\mathfrak{U}} %
\newcommand{\gotb}{\mathfrak{b}} %
\newcommand{\trnsp}[1]{{#1}^{\text{Т}}}
\newcommand{\const}{\rm const}
\newcommand{\br}[1]{\left(#1\right)}
\newcommand{\tr}[1]{\left\langle #1\right\rangle}
\newcommand{\secpar}{\lambda}

\newcommand{\FF}{\mathcal{F}}
\newcommand{\KK}{\mathcal{K}}

\DeclareMathOperator{\KDF}{\mathtt{KDF}}
\DeclareMathOperator{\mods}{mods}

\newcommand{\cert}{\mathtt{Cert}}
\newcommand{\sign}{\mathtt{SGN}}
\newcommand{\verify}{\mathtt{VERIFY}}
\DeclareMathOperator{\prf}{\mathtt{prf}}
\DeclareMathOperator{\mac}{\mathtt{MAC}}
\DeclareMathOperator{\encrypt}{\mathtt{enc}}
\DeclareMathOperator{\decrypt}{\mathtt{dec}}
\DeclareMathOperator{\kdf}{\mathtt{KDF}}
\newcommand{\id}{\ensuremath{\mathtt{ID}}}
\newcommand{\ID}{\ensuremath{\mathtt{ID}}}
\newcommand{\adv}{\ensuremath{\mathcal{M}}}
\newcommand{\OI}{\ensuremath{\mathtt{OI}}}


\newtheorem{theorem}{Теорема}
\newtheorem{lemma}{Лемма}
\newtheorem{predl}{Предложение}
\newtheorem{sled}{Следствие}
\newtheorem{utver}{Утверждение}
\theoremstyle{definition}
\newtheorem{remark}{Замечание}
\newtheorem{task}{Задачи}
%\newtheorem{algorithm}{Алгоритм}
\newtheorem{definition}{Определение}
\newtheorem{example}{Пример}
\renewcommand{\proofname}{Доказательство}
\newtheorem{hypo}{Гипотеза}
\newtheorem{problem}{Задача}
\renewcommand{\abstractname}{ABstract}

\setromanfont{Times New Roman}
\setsansfont{Arial}
\setmonofont{Courier}
%\setallmainfonts(Cyrillic){Liberation Serif}


\newcommand{\mallory}{\ensuremath{\mathfrak{M}} }

%\textwidth=160mm \headheight=0mm \topmargin=0mm \textheight=215mm 

\usepackage{fancyhdr}
\patchcmd{\chapter}{plain}{empty}{}{}
\renewcommand{\thesection}{\arabic{section}}
\renewcommand{\thesubsection}{\arabic{section}.\arabic{subsection}}
\renewcommand{\thesubsubsection}{\arabic{section}.\arabic{subsection}.\arabic{subsubsection}}
\renewcommand{\baselinestretch}{1.35}
\frenchspacing \sloppy \emergencystretch=5pt\textwidth=170mm \textheight=257mm
\oddsidemargin=4.6mm \headheight=10mm \footskip=10mm \headsep=10mm \topmargin=-1in \makeatletter
\renewcommand{\@oddhead}{\ifnum\thepage>1{\hfil\large\thepage\hfil}\fi}
\renewcommand{\@evenhead}{\ifnum\thepage>1{\hfil\large\thepage\hfil}\fi}
\pagestyle{fancy}
\fancyhf
\lhead{}
\chead{\ifnum\thepage>1{\hfil\large\thepage\hfil}\fi}
\rhead{}
\lfoot{}
\cfoot{}
\rfoot{}
\renewcommand{\headrulewidth}{0pt}
\renewcommand{\footrulewidth}{0pt}

\renewcommand{\Authands}{, }
 
\title{ Протокол выработки общего ключа на основе аппарата изогений суперсингулярных эллиптических кривых \\(проект)}
\author[1]{С.~Гребнев} 
\author[2]{П.~Ключарёв}
\author[3]{А.~Коренева}
\author[4]{Д.~Кошелев}
\author[5]{О.~Тараскин}
\author[3]{А.~Тулебаев}
\affil[1]{QAPP}
\affil[2]{МГТУ им.~Н.Э.~Баумана}
\affil[3]{Код безопасности}
\affil[4]{ИнфоТеКС}
\affil[5]{Waves}

\date{Москва, 2020}
 
\begin{document} 

\maketitle
\newpage
\tableofcontents
\newpage

\renewcommand{\baselinestretch}{1.35}\large
\section*{Введение}\addcontentsline{toc}{section}{Введение}

В настоящем  отчете излагаются  результаты исследований возможностей создания протокола выработки общего ключа   на основе механизма изогений суперсингулярных эллиптических кривых. 
%Предлагаемый протокол основан на схеме де Фео-Яо-Плута \cite{DF}.

Дадим краткую характеристику предлагаемой схемы.

 \begin{itemize}
  \item Схема -- типа SIDH \cite{DF}.
  \item Только эфемерные ключи.
  \item Три уровня квантовой стойкости --- 80, 128 и 256 битов
  \item Стартовая кривая специального вида $E_{19}$.
  \item Представление кривых в форме Монтгомери. 
  \item Отказ от компрессии открытых ключей в пользу быстродействия.
 \end{itemize}


\section{Основные определения}

В данном разделе кратко приводятся основные сведения о математическом аппарате изогений эллиптических кривых, более подробно см.  монографии \cite{Silverman, Wash}.

\subsection{Базовые определения}

Рассмотрим эллиптическую кривую над  полем $K$, $\Char K\neq2,3$, заданную в \emph{краткой форме Вейерштрасса}: $E_{a,b}(K): y^2=x^3+ax+b$.

\begin{definition}
\emph{Кольцо регулярных функций} на кривой: факторкольцо $$\overline{K}[E_{a,b}]=\overline{K}[x,y]/(y^2-x^3-ax-b).$$
Так как $(y^2-x^3-ax-b)$ неприводим, то в кольце регулярных функций нет делителей нуля.
Его поле частных $\overline{K}_{a,b}(x,y)$ -- \emph{поле рациональных функций} на $E_{a,b}$.
\end{definition}

%\begin{definition}
%Пусть $f(x,y)\in\overline{K}(E_{a,b})$, $(x_0,y_0)\in %E_{a,b}(\overline{K})$ -- класс рациональных функций из поля %$\overline{K}(x,y)$. Скажем, что  $f(x,y)$
%\emph{определена в точке} $(x_0,y_0)$, если найдется рациональная функция из $\overline{K}(x,y)$, содержащаяся в классе $f(x,y)$ и такая, что ее знаменатель не обращается в $0$ в $(x_0,y_0)$. Тогда определено ее значение, которое называется значением функции $f(x,y)$ в точке $(x_0,y_0)$
%(определение, очевидно, корректно).
%\end{definition}

%Аналогично в проективном пространстве -- рассматриваем однородные многочлены.

\begin{definition}
Пусть $E_{a,b}$, $E_{a_1,b_1}$ -- эллиптические кривые над $K$. \emph{Рациональное отображение} (\emph{морфизм}) $E_{a,b}$ в $E_{a_1,b_1}$ -- отображение вида
$$
\psi=\psi(x,y)=(f_1(x,y),f_2(x,y)),
$$
где $f_1(x,y),f_2(x,y)\in\overline{K}(E_{a,b})$, такое, что для любой точки $(x_0,y_0)\in E_{a,b}$ такой, где функции определены,
верно $(f_1(x_0,y_0),f_2(x_0,y_0))\in E_{a_1,b_1}.$
\end{definition}

%\begin{definition}
%Если рациональное отображение определено в каждой точке %$E_{a,b}(K)$, то это \emph{морфизм}.
%\end{definition}

\begin{definition}
Если $\psi$ -- рациональное отображение и $\psi(\mathcal{O})=\mathcal{O}_1$, то $\psi$ -- \emph{изогения}.
% (проверить, что $\psi(\mathcal{O})=\mathcal{O}_1$, можно в проективном пр-ве, где нули э.к. -- обычные точки). 
Если такое отображение существует, то соответствующие кривые называются \emph{изогенными}.
\end{definition}

\begin{theorem} (Тэйт).
Две кривые над  конечным полем $K$ изогенны тогда и только тогда, когда порядки их групп равны.
\end{theorem}

Согласно \cite[2.9]{Wash}, рациональное отображение $\psi(x,y)$, $\psi:E_{a,b}\to E_{a_1,b_1}$ 
может быть записано в канонической форме $\psi(x,y)=(r_1(x),yr_2(x))$, где $r_1,r_2$ -- рациональные функции. Под \emph{степенью} изогении будем понимать степень $r_1$ как рациональной функции. Изогении степени $l$ будем также называть $l$-изогениями. 

\begin{definition}
Изогения называется \emph{сепарабельной}, если $r_1'(x)\neq 0$.
\end{definition}

Далее рассматриваются только сепарабельные изогении.

Известно, что  для любой  изогении $ \psi:E\to E'$   существует единственная \emph{дуальная} изогения $\hat{\psi}:E'\to E$ такая, что $\hat{\psi}\circ \psi=[m]_E$ и $\psi\circ\hat{\psi}=[m]_{E'}$, где $m$ -- {степень} изогении.

Если рассмотреть три эллиптические кривые $E,E',E''$ и изогении $\phi,\psi$: $\phi:E\mapsto E'$, $\psi:E'\mapsto E''$,
то определены \emph{композиции} изогений $\psi\circ\phi:E\mapsto E''$, и $\widehat{\psi\circ\phi}=\hat{\phi}\circ\hat{\psi}$, а также $\deg \psi\circ\phi=\deg\psi \deg\phi$.

С практической точки зрения под изогенией $\varphi$ достаточно понимать пару рациональных функций
$$
\frac{N(x)}{D(x)}=\frac{x^n+\cdots+n_1x+n_0}{x^{n-1}+\cdots+d_1x+d_0},
$$
где $D(x)$ обращается в $0$ на ядре изогении $\ker(\varphi)$.

\begin{definition}
\emph{Графом изогений} называется граф, множеством вершин которого является множество классов изоморфизма эллиптических кривых. 
Две различных вершины этого графа соединены ребром тогда и только тогда, когда представители соответствующих классов изоморфизма изогенны.
\end{definition}

\subsection{Суперсингулярные эллиптические кривые}

\begin{definition}
Морфизм эллиптической кривой в себя -- \emph{эндоморфизм}.
Эндоморфизмы эллиптической кривой $E(K)$, определенной над полем $K$, образуют кольцо относительно операций сложения (поточечного) и композиции, оно обозначается $End(E)$. 
\end{definition}

\begin{example}
\begin{enumerate}

\item $E(_{a,b}(GF(p))$, \emph{эндоморфизм Фробениуса}: $(x,y)\mapsto (x^p,y^p)$

\item $E_{a,b}(K)$, $P=(x,y)\mapsto -P=(x,-y)$ -- эндоморфизм

\item $E_{a,b}\to E_{a,b}, P\mapsto mP$ -- изогения (и эндоморфизм), обозначается $[m]$.
\end{enumerate}
\end{example}

Пусть $K=GF(q)$, $\Char (K)=p$. Для $m\in\NN$ обозначим $E_{a,b}[m]$ множество точек $(x,y)\in E_{a,b}(\overline{K})$ таких, что $mP=\mathcal{O}$ -- \emph{группу кручения} кривой $E_{a,b}$. 

\begin{theorem}\label{thx}
 \begin{enumerate}
  \item Если $(m,p)=1$, то  $E_{a,b}[m]$ изоморфна $\ZZ/m\ZZ\times \ZZ/m\ZZ$.
  \item Группа $E_{a,b}[p^e] $ либо равна $\{\mathcal{O}\}$ для всех $e=1,2,\dots$, %(тогда кривая суперсингулярна), 
  либо для всех $e$  изоморфна $\ZZ/p^e\ZZ$.  
 \end{enumerate}
\end{theorem}

\begin{definition}
В случае, если группа $E_{a,b}[p^e] $  равна $\{\mathcal{O}\}$ для всех $e=1,2,\dots$, кривая называется \emph{суперсингулярной}.
\end{definition}

\begin{definition}
Если кольцо эндоморфизмов $End(E)$, рассматриваемое как $\ZZ$-модуль, имеет ранг 4, то кривая  $E(K)$ называется \emph{суперсингулярной}. 
\end{definition}


Для случая $GF(p)$ суперсингулярные кривые определяются условием $\#E=p+1$. %Отсюда следует, что MOV-степень суперсингулярной кривой относительно любого простого делителя числа $\#E$ равна 2, и задача дискретного логарифмирования в указанной группе сводится к дискретному логарифмированию в $GF(p^2)$. 
%Поэтому суперсингулярные кривые имеют репутацию <<слабых>> с точки зрения использования в схемах с открытым ключом даже при больших~$p$. 

%\begin{example}Рассмотрим случаи

 %$p\equiv2\bmod 3,\,y^2=x^3+a$; $p=3\bmod4,\,y^2=x^3+x$. %Для этих кривых существует более быстрый вариант алгоритма Миллера \cite{bKLS02}.
%\end{example}

Все суперсингулярные кривые определены над $GF(p^2)$, их количество приблизительно равно $p/12$.

Граф $l$-изогений при простом $l\neq p$ между суперсингулярными эллиптическими кривыми обладает следующими свойствами \cite{Piz1, Piz2}:
\begin{itemize}
 \item он $(l+1)$-регулярный;
 \item содержит единственную связную компоненту, состоящую из всех суперсингулярных эллиптических кривых;%  с одинаковым количеством точек;
\item экспандер (конечный ненаправленный мультиграф, в котором любое подмножество вершин, не являясь <<слишком большим>>, имеет <<сильную>> связность).
 \end{itemize}


\subsection{Вычисление изогений}

Для построения изогений заданной степени можно воспользоваться формулами Велю \cite{Velu}. 

Итак, пусть имеется эллиптическая кривая $E_{a,b}(K)$, заданная в форме Вейерштрасса ($y^2=x^3+ax+b$) над полем $K$ характеристики, отличной от 2 и 3. Пусть $F$ -- подгруппа $E_{a,b}(K)$ порядка $l$. Тогда изогения с ядром $F$ строится по следующему алгоритму.

\begin{enumerate}
 \item Разобьем $F\setminus\{\mathcal{O}\}$ на три непересекающихся множества, $F=F_2\cup R_+\cup R_-$, где $F_2$ -- множество точек четного порядка,
 а $R_+$ и $R_-$ -- разбиение множества точек нечетного порядка так, что $R\in R_+$ тогда и только тогда, когда $-R\in R_-$.
 \item Определим множество $S$: $S=F_2\cup R_+$.
 \item Для каждой точки $Q\in S$ будем вычислять
 $$
 g_Q^x=3x^2x_Q+a, g_Q^y=-2y_Q
 $$
 (здесь $(x_Q,y_Q)$ -- координаты точки $Q$); 
 если $Q=-Q$, то $v_Q=g_Q^x$, иначе $v_Q=2g_Q^x$;
 $$
 u_Q=(g_Q^x)^2
  $$
  \item Вычислим
  $$
  v=\sum_{Q\in S} v_Q;
  $$
  $$
  w=\sum_{Q\in S} (u_Q+x_Qv_Q).
  $$
\item  Коэффициенты изогенной кривой определяются как 
$$
a'=a-5v;
$$
$$
b'=b-7w.
$$
\item Формулы преобразования координат $(x,y)\mapsto (x',y')$ имеют вид
$$
x'=x+\sum_{Q\in S} \left(\frac{v_Q}{x-x_Q}+\frac{u_Q}{(x-x_Q)^2}\right),
$$

$$
y'=y+\sum_{Q\in S} \left(u_Q\frac{2y}{(x-x_Q)^3}+v_Q\frac{y-y_Q}{(x-x_Q)^2}-\frac{g_Q^xg_Q^y}{(x-x_Q)^2}\right).
$$
  \end{enumerate}

  Трудоемкость алгоритма Велю пропорциональна степени изогении $l$.
  
Для случая подгруппы $F$, имеющей порядок $l^e$, можно вычислить соответствующую изогению как композицию $e$ изогений степени $l$:
пусть $F=\tr{G}$, тогда $\varphi=\varphi_{l-1}\circ\varphi_{l-2}\circ\cdots\circ \varphi_0$, 
$$\varphi_0:E\to E_1=E/\tr{l^{e-1}G}, G_1=\varphi_0(G);$$
$$\varphi_i:E_i\to E_{i+1}=E_i/\tr{l^{e-i-1}G_i}, G_{i+1}=\varphi_i(G_i). $$
  
Учитывая сказанное, приведем формулы Велю для случаев $l=2,3$.

 \subsubsection*{Случай $l=2$}
Пусть  $P=(X_P,y_P)$ -- точка порядка 2 на $E_{a,b}(GF(p^2))$. Положим $v=3X_P^2+a$, $a'=a-5v$, $b'=b-7vX_P$;
тогда $E_{a',b'}(GF(p^2)): Y^2=X^3+a'X+b'$ -- 2-изогенная $E_{a,b}$ кривая, и  отображение
 $$
 (x,y)\mapsto \left(\frac{v}{x-X_P}+X_P, Y-\frac{vY}{(X-X_P)^2}\right) 
 $$
 задает 2-изогению из $E_{a,b}$ в $E_{a',b'}$ с ядром $\tr{P}$. 

   
 \subsubsection*{Случай $l=3$}
 Пусть $P=(X_P,y_P)$ -- точка порядка 3 на $E_{a,b}(GF(p^2))$. Положим $v=3X_P^2+a$, $u=4Y_P^3$, $a'=a-5v$, $b'=b-7(u+vX_P)$;
 тогда $E_{a',b'}(GF(p^2)): Y^2=X^3+a'X+b'$ -- 3-изогенная $E_{a,b}$ кривая, и
 отображение
 $$
 (x,y)\mapsto \left(\frac{v}{x-X_P}+\frac{u}{(X-X_P)^2}+X_P, Y\left(1-\frac{v}{(X-X_P)^2}+\frac{2u}{(X-X_P)^3}\right)\right) 
 $$
 задает 3-изогению из $E_{a,b}$ в $E_{a',b'}$ с ядром $\tr{P}$. 

  
  
\subsection{Изогении между суперсингулярными кривыми}

Напомним, что у двух изоморфных кривых одинаковый $j$-инвариант.
Из-за того, что построение изоморфизма между кривыми особой сложности не представляет, 
задача  изогений по сути является задачей про нахождение изогений между различными классами изоморфных кривых (а каждый из этих классов можно представить  соответствующим $j$-инвариантом).

%Первоначально криптографические схемы предлагалось строить на обыкновенных кривых (см. \cite{RoSt}), но позже в этом случае была продемонстрирована эффективная квантовая атака \cite{AntiRoSt}. Поэтому в схемах де Фео-Яо-Плута используются суперсингулярные кривые.

\subsection{Кривые Монтгомери}
 

Математический аппарат эллиптических кривых в форме Монтгомери \cite{Mon} показал себя наиболее эффективным
при реализации схем, основанных на изогениях, поэтому в настоящем разделе мы кратко напомним некоторые их свойства.

\emph{Кривая Монтгомери} задается уравнением 
\begin{equation}\label{monty}
M_{A,B}(GF(p)): By^2=x^3+Ax^2+x, \text{ где } A,B\in GF(p), B(A^2-4)\neq 0.
\end{equation}

Выполним замену координат $(x,y)\mapsto\left(\frac{x}{B}+\frac{A}{3B},\frac{y}{B}\right)$, положим  $a=1/B^2-A^2/(3B^2)$, $b=-A^3/(27B^3)-aA/(3B)$ 
-- получим преобразование к краткой форме Вейерштрасса $E_{a,b}$.

\begin{theorem} Эллиптическая кривая, заданная в краткой форме Вейерштрасса $E_{a,b}$,
 может быть преобразована в форму Монтгомери тогда и только тогда, когда многочлен $x^3+ax+b$ 
имеет корень $\alpha$ в $GF(p)$ и $(3\alpha^2+a)$ есть квадратичный вычет $mod p$.
\end{theorem}

Если условия этой теоремы выполнены, то положим $A=3\alpha s$, $B=s$, где $s$ -- квадратный корень из $(3\alpha^2+a)^{-1}$, 
и определим замену координат по формуле $(x,y)\mapsto (x/s+\alpha,y/s)$.

На таких кривых точка $(0,0)$ имеет порядок 2, и $\#M_{A,B}$ делится на 4. 

$j$-инвариант кривой Монтгомери вычисляется по формуле
\begin{equation}
j(M_{A,B})=\frac{256(A^2-3)^3}{A^2-4}.
\end{equation}

Операции на кривых Монтгомери задаются следующими формулами:
\small
\begin{multline*}
(x_1,y_1)+(x_2,y_2)=\left(B\frac{(y_2-y_1)^2}{(x_2-x_1)^2}-A-x_1-x_2, \frac{(2x_1+x_2+A)(y_2-y_1)}{(x_2-x_1)}-B\frac{(y_2-y_1)^3}{(x_2-x_1)^3-y_1)}\right)
\end{multline*}
при $(x_1,y_1)\neq(x_2,-y_2)$;
\begin{multline*}
[2](x_1,y_1)= \left(\frac{(x_1^2-1)^2}{4x_1(x_1^2+Ax_1+1)} , y_1\frac{(x_1^2-1)(x_1^4+2Ax_1^3+6x_1^2+2Ax_1+1)}{8x_1^2(x_1^2+Ax_1+1)^2}\right);
\end{multline*}
\begin{multline*}
[3](x_1,y_1)= \left(\frac{(x_1^4-4Ax_1-6x_1^2-3)^2x_1}{(3x_1^4+4Ax_1^3+6x_1^2-1)^2}\right., \\
\left.y_1\frac{(x_1^4-4Ax_1-6x_1^2-3)(x_1^8+4Ax_1^7+28x_1^6+28x_1^5+(16A^2+6)x_1^4+28Ax_1^3+28x_1^2+4AX_1+1))}{(4Ax_1^3+3x_1^4+6x_1^2-1)^3}\right)
\end{multline*}
\large


Точка $(x,y)$ на кривой Монтгомери (\ref{monty}) представляется в \emph{проективных координатах} в виде пары $(X:Z)$ такой, что $x=X/Z$.
Подобное представление позволяет использовать эффективные алгоритмы сложения и удвоения \cite{EFD,Mon}, вычисляющие только $X$-координату точки, 
что в ряде случаев %(но не в случае схемы цифровой подписи!) 
является достаточным.

Пусть $m>n>0$, $P=(X_1:Y_1:Z_1)\in M_{A,b}$, известны кратные точки $P_{n}=[n]P$, $P_{m}=[m]P$, $P_{m-n}=[(m-n)]P$. Тогда имеют место формулы
\begin{eqnarray*}
X_{m+n}=Z_{m-n}((X_m-Z_m)(X_n+Z_n)+(X_m+Z_m)(X_n-Z_n))^2\\
Z_{m+n}=X_{m-n}((X_m-Z_m)(X_n+Z_n)+(X_m+Z_m)(X_n-Z_n))^2
\end{eqnarray*}
\begin{eqnarray*}
4X_nZ_n=(X_n+Z_n)^2-(X_n-Z_n)^2\\
X_{2n}=(X_n+Z_n)^2(X_n-Z_n)^2\\
Z_{2n}=4X_nZ_n((X_n-Z_n)^2+((A+2)/4)(4X_nZ_n))
\end{eqnarray*}

\subsection{Вычисление изогений на кривых Монтгомери}

TODO!!!

\subsection{Арифметика в $GF(p^2)$}

Пусть $p\equiv 3\pmod{4}$, тогда $GF(p^2) = GF(p)(\imath)$, где $\imath^2+1=0$.
Элементы поля представляются в виде $u=u_0+u_1\cdot\imath$, $u_0, u_1\in GF(p)$.
Операции в поле $GF(p^2)$ задаются естественным образом.
\begin{itemize}
 \item Сложение: $u+v = (u_0+v_0) + (u_1+v_1)\cdot\imath$.
 \item Аддитивная инверсия: $-u=-u_0+(-u_1)\cdot\imath$
 \item Умножение: $uv = (u_0v_0-u_1v_1) + (u_0v_1+u_1v_0)\cdot\imath$.
 \item Мультипликативная инверсия: $u^{-1} = (u_0)(u_0^2+u_1^2)^{-1} + (-u_1)(u_0^2+u_1^2)^{-1}\cdot\imath   $.
\end{itemize}




\section{Протокол выработки общего ключа}\label{protodesc}

В протоколе принимают участие два абонента: \emph{инициатор} $A$ и \emph{ответчик} $B$.

\subsection*{Параметры протокола}

Параметрами протокола являются:
\begin{itemize}
 \item простое число $p$ вида $ p=l_A^{e_A}l_B^{e_B}\cdot f  - 1$, где $l_A,l_B$ -- маленькие простые (например, 2 и 3),
$(l_A,f)=(l_B,f)=1$;
\item поле $GF(p^2)$;
\item суперсингулярная кривая $E_0(GF(p^2))$, мощность группы точек которой равна $(l_A^{e_A}l_B^{e_B}\cdot f)^2$ (\emph{стартовая кривая}). По построению (см. \cite{DF}), $E[l_A^{e_A}]$ содержит $l_A^{e_A-1}(l_A+1)$ циклических подгрупп порядка $l_A^{e_A}$, каждая из которых определяет собственную изогению (т.е. ядром которой она является), аналогичное замечание  верно и для $E[l_B^{e_B}]$.
\end{itemize}

В основе протокола лежит следующая коммутативная диаграмма:
\begin{equation}\label{comd}
\begin{CD}
 E @>\varphi >> E/\tr{P} \\
 @V\psi VV @VVV \\
 E/\tr{Q} @>>> E/\tr{P,Q}
\end{CD}
\end{equation}
где $\varphi, \psi$ -- случайные пути в графах изогений степеней $l_A$, $l_B$ соответственно. Протокол представляет собой вариант схемы Диффи-Хеллмана, реализованный над диаграммой (\ref{comd}). Идея его состоит в том, чтобы абонент $A$ выбирал $\varphi$, а $B$ выбрал $\psi$. Стойкость протокола основана на 
сложности нахождения пути, соединяющего две вершины в графе. 

\subsection*{Протокол выработки общего ключа}

%Этот протокол представляет собой вариант схемы Диффи-Хеллмана, реализованный над диаграммой (\ref{comd}). Идея его состоит в том, чтобы абонент $A$ выбирал $\varphi$, а $B$ выбрал $\psi$.

%%FIXME!!!

Фиксируем открытые параметры протокола:
\begin{itemize}
 \item простое число $  p=l_A^{e_A}l_B^{e_B}\cdot f  - 1$, где $l_A,l_B$ -- маленькие простые (например, 2 и 3),
$(l_A,f)=(l_B,f)=1$;
 \item стартовую кривую $E_0(GF(p^2))$;
 \item базисы $\{P_A,Q_A\}$ и $\{P_B,Q_B\}$, которые порождают, соответственно, $E_0[l_A^{e_A}]$ и $E_0[l_B^{e_B}]$, т.е. $\tr{P_A,Q_A}=E_0[l_A^{e_A}]$ и $\tr{P_B,Q_B}=E_0[l_B^{e_B}]$.

\end{itemize}

 Абонент $A$ выбирает  случайный элемент $ n_A\in_R\ZZ/l_A^{e_A}\ZZ$ %,% который одновременно не делится на $l_A$, 
 и строит изогению $\varphi_A:E_0\to E_A$ с ядром $K_A:=\tr{ P_A+[n_A]Q_A}$.  Абонент $A$  также вычисляет образ $\{\varphi_A(P_B),\varphi_A(Q_B)\}$ и посылает эти точки абоненту $B$ вместе с эллиптической кривой $E_A$ (т.е. ее описанием).
 
 Аналогично, абонент $B$ выбирает   случайный элемент  $ n_B\in_R\ZZ/l_B^{e_B}\ZZ$ %, %который одновременно не делится на $l_B$, 
 и строит изогению $\varphi_B:E_0\to E_B$ с ядром $K_B:=\tr{ P_B+[n_B]Q_B}$.  Абонент $B$  также вычисляет образ $\{\varphi_B(P_A),\varphi_B(Q_A)\}$ и посылает эти точки абоненту $A$ вместе с эллиптической кривой $E_B$ (т.е. ее описанием).
 
 Получив от абонента $B$ набор $E_B, \varphi_B(P_A),\varphi_B(Q_A)$, абонент $A$ строит изогению $\varphi'_A:E_B\to E_{AB}$ с ядром $\tr{ \varphi_B(P_A)+[n_A]\varphi_B(Q_A)}$; абонент $b$ выполняет аналогичные действия.
 В качестве общего ключа  используется $j$-инвариант кривой
 $$
 E_{AB}=\varphi'_B(\varphi_A(E_0))=\varphi'_A(\varphi_B(E_0))=E_0/\tr{ P_A+[n_A]Q_A,  P_B+[n_B]Q_B }.
 $$

\section{Анализ стойкости}\label{analysis}

В рамках этого раздела считаем, что $p=l_A^{e_A}\cdot l_B^{e_B}\cdot f-1$ -- сбалансированное простое (т.е. $l_A^{e_A}\approx l_B^{e_B}\approx p^{1/2}$), $E_0(GF(p^2))$ -- суперсингулярная эллиптическая кривая, базисы $\{P_A,Q_A\}$ и $\{P_B,Q_B\}$
порождают, соответственно, $E_0[l_A^{e_A}]$ и $E_0[l_B^{e_B}]$, т.е. $\tr{P_A,Q_A}=E_0[l_A^{e_A}]$ и $\tr{P_B,Q_B}=E_0[l_B^{e_B}]$. 

Иногда, для краткости, мы будем опускать индексы $A, b$ и использовать обозначения $l^e$.

\subsection{Формальный анализ}

Основная вычислительная задача, на предположение о сложности которой опирается стойкость предлагаемого протокола, состоит в следующем \cite{DF}.

\begin{problem}\label{problem31} \emph{Вычислительная задача суперсингулярных изогений, Computational Supersingular Isogeny --- CSSI}: пусть $\phi_1: E_0 \to E_1$ -- изогения с ядром
$R_1 + [n]S_1$, где   $n $ выбрано случайно равновероятно из $\ZZ/l^{e}\ZZ$. По $E_1$ и значениям образов $\phi_1(R_2), \phi_1(S_2)$ найти порождающий элемент группы $\tr{R_1+[n]S_1}$.
\end{problem}

Можно сформулировать и аналог вычислительной задачи Диффи--Хеллмана для изогений суперсингулярных эллиптических кривых.

\begin{problem}\label{problem32} \emph{Вычислительная задача Диффи--Хеллмана для суперсингулярных изогений, Supersingular Computational Diffie--Hellman ---  SSCDH}: 
пусть $\phi_A: E_0 \to E_A$ -- изогения с ядром $\tr{P_A+[n_A]Q_A}$, где   $n_A $ выбрано случайно равновероятно из $\ZZ/l_A^{e_A}\ZZ$,
и пусть $\phi_B: E_0 \to E_B$ -- изогения с ядром $\tr{P_B+[n_B]Q_B}$, где   $n_B $ выбрано случайно равновероятно из $\ZZ/l_B^{e_B}\ZZ$,
По $E_A,E_B$ и значениям образов $\phi_A(P_B), \phi_A(Q_B), \phi_B(P_A), \phi_B(Q_A)$ найти порождающий элемент группы $\tr{P_A+[n_A]Q_A, P_B+[n_B]Q_B}$.
\end{problem}

Далее, сформулируем аналог распознавательного варианта задачи Диффи--Хеллмана: 

 \begin{problem}\label{problem8} \emph{Распознавательная задача суперсингулярных изогений, Supersingular Decisonal Diffie--Hellman --- SSDDH}: пусть задан набор , выбранный с вероятностью $1/2$ из одного из следующих распределений
\begin{itemize}
\item $(E_A,E_B,\phi_A(P_B), \phi_A(Q_B), \phi_B(P_A), \phi_B(Q_A),E_{AB})$, где\\ $(E_A,E_B,\phi_A(P_B), \phi_A(Q_B), \phi_B(P_A), \phi_B(Q_A))$ -- как в предыдущем определении,
$$
E_{AB} \cong E_0/\tr{P_A+[n]Q_A, P_B+[n]Q_B};
$$
\item $(E_A,E_B,\phi_A(P_B), \phi_A(Q_B), \phi_B(P_A), \phi_B(Q_A),E_{C})$, где\\  $(E_A,E_B,\phi_A(P_B), \phi_A(Q_B), \phi_B(P_A), \phi_B(Q_A))$ --  как в предыдущем определении,  
$$
E_C \cong E_0/\tr{P_A+[n']Q_A, P_B+[n']Q_B}
$$
где $ n'$ выбрано случайно из $\ZZ/l_B^{e_B}\ZZ$);
\end{itemize}
определить, из какого распределения выбран указанный набор. 
\end{problem}

Редукция задачи SSCDH к SSCI очевидна, в обратном направлении -- неизвестна, как и в случае с задачей Диффи--Хеллмана и дискретного логарифмирования в абелевой группе.

Далее, напомним, следуя \cite{DF}, основные положения модели Канетти--Кравчика из \cite{CK}.

Рассмотрим конечное множество абонентов $P_1,\dots,P_n$, моделируемых вероятностными алгоритмами. Противник $\mathcal{I}$, который также моделируется вероятностным алгоритмом,
контролирует все коммуникации между абонентами (с тем исключением, что противник не может вставлять или модифицировать сообщения -- кроме сообщений от нечестных абонентов).
Любое сообщение может быть доставлено лишь один раз. Абоненты передают исходящие сообщения противнику, который контролирует их доставку, через запрос $\mathtt{Send}$. Абоненты
активируются запросом $\mathtt{Send}$, и таким образом противник контролирует создание сеансов протокола. Два сеанса $s$ и $s'$ являются \emph{соответствующими (matching)}, если
исходящие сообщения одного сеанса являются входящими для другого, и наоборот.

Противник получает доступ к запросам $\mathtt{SessionStateReveal}$, $\mathtt{SessionKeyReveal}$, $\mathtt{Corrupt}$. 

Запрос $\mathtt{SessionStateReveal}(\mathfrak{s})$ позволяет противнику получить состояние текущего сеанса $\mathfrak{s}$, в том числе любую секретную информацию. Запрос фиксируется, и сеанс 
не выдает выходной информации.

Запрос  $\mathtt{SessionKeyReveal}(\mathfrak{s})$ выдает противнику сеансовый ключ для заданного сеанса $\mathfrak{s}$.

Запрос $\mathtt{Corrupt}(P_i)$ передает противнику контроль над абонентом $P_i$, в том числе всю информацию в памяти абонента, включая сохраненные сеансовые ключи и другую информацию о сеансах. Захваченный абонент не производит более выходной информации.

Назовем сеанс $\mathfrak{s}$  с владельцем $P_i$ \emph{локально раскрытым}, если противник выполнил один из запросов $\mathtt{SessionStateReveal}(\mathfrak{s})$, $\mathtt{SessionKeyReveal}(\mathfrak{s})$, $\mathtt{Corrupt}(P_i)$, до того, как сеанс завершился. Скажем, что сеанс \emph{раскрыт}, если локально раскрыт сеанс или соответствующий ему, иначе назовем сеанс \emph{свежим}.

Далее, противнику $\mathcal{I}$ разрешается единственный запрос $\mathtt{Test}(\mathfrak{s})$, который может быть применен на любом этапе к завершенному свежему сеансу $\mathfrak{s}$.
Выбирается случайный бит $b$. Если $b=0$, то оракул раскрывает сеансовый ключ, иначе, если $b=1$, оракул вырабатывает случайное значение из пространства сеансовых ключей. Затем противник может выполнять любые запросы, за исключением того, что он не может пытаться раскрыть тестовый сеанс $\mathfrak{s}$. В любой момент противник может попытаться угадать $b$. Обозначим
$\mathrm{GoodGuess}^{\mathcal{I}}(k)$ событие, состоящее в том, что $\mathcal{I}$ угадал $b$, и определим \emph{преимущество}
$$
\mathrm{Adv}^{\mathcal{I}}(k)=\mathrm{max}\left\{0,\left|\mathrm{Pr}[\mathrm{GoodGuess}^{\mathcal{I}}(k)]-\frac{1}{2}\right|\right\},
$$
где $k$ -- параметр стойкости.

\begin{definition}\label{defstoi}
 Протокол выработки общего ключа $\Pi$ с параметром стойкости $k$ является стойким относительно задачи определения сеансового ключа в модели Канетти--Кравчика, если выполнено следующее:
 \begin{enumerate} 
  \item если два честных абонента завершили совпадающие сеансы, то соответствующие сеансовые ключи совпадают; 
  \item $\mathrm{Adv}^{\mathcal{I}}(k)$ -- пренебрежимо малая величина.
 \end{enumerate}

\end{definition}

\begin{theorem} В предположении о сложности задачи SSDDH, протокол из раздела \ref{protodesc} является стойким относительно задачи определения сеансового ключа в модели Канетти--Кравчика.
 \end{theorem}
\begin{proof}
 Поскольку наш протокол является вариантом протокола из \cite{DF}, то доказательство в точности повторяет рассуждения из \cite[Theorem 6.1]{DF}. Для полноты изложения приведем эти рассуждения.

Два честных абонента в совпадающих сеансах вырабатывают одинаковый общий ключ, поэтому первая часть Определения \ref{defstoi} выполнена.

Для доказательства второй части предположим, что существует полиномиальный противник $\mathcal{I}$ с преимуществом $\epsilon$, не являющимся пренебрежимо малой величиной.
Тогда Алгоритм \ref{algssddh} является полиномиальным различителем для задачи SSDDH.

\begin{algorithm}
\caption{Различитель для SSDDH}
 \label{algssddh}
\textbf{Input}:   $E_A, E_B, \phi_A(P_B),\phi_A{Q_B},\phi_B(P_A),\phi_B(Q_A)$\\
%\textbf{Output}:  $d$
 \begin{algorithmic}[1]
  \State $r\stackrel{R}{\gets}\{1,\dots,k\}$, где $k$ -- максимальное количество активируемых $\mathcal{I}$ сеансов
  \State Вызвать $\mathcal{I}$ и симулировать протокол для $\mathcal{I}$б за исключением $r$-го активированного сеанса
  \State Для $r$-го сеанса: $A$ отправляет $A,i,E_A,\phi_A(P_B),\phi_A(Q_B)$, а $B$ отправляет $B,i,E_B,\phi_B(P_A),\phi_A(P_B)$, где $i$ -- идентификатор сеанса
  \If{$r$-й сеанс выбран $\mathcal{I}$ в качестве тестового сеанса}
  \State Установить $\mathcal{I}$ как ответ на тестовый запрос
  \State $d\gets $ вывод $\mathcal{I}$
  \Else 
  \State $d\stackrel{R}{\gets}\{0,1\}$
  \EndIf
  \State \Return d
 \end{algorithmic}

\end{algorithm}

Полиномиальность алгоритма очевидна, осталось доказать, что он имеет преимущество, не являющееся пренебрежимо малой величиной.
Если $r$-й сеанс не является тестовым, то Алгоритм \ref{algssddh} выдает случайный бит и, таким образом, его преимущество равно $0$.
Иначе, $\mathcal{I}$  успешно завершится с вероятностью $\epsilon$, поскольку симулированный протокол для $\mathcal{I}$  неотличим от действительного.
Последний случай имеет место с вероятностью $1/k$, следовательно, преимущество различителя для задачи SSDDH равно $\epsilon/k$, что не является пренебрежимо малой величиной.
 

 
 
 \end{proof}



\subsection{Практический анализ}

Дадим несколько упрощенную формулировку SSCI, достаточную для наших целей. Итак, пусть задана (секретная) $l^e$-изогения $\phi:E_0\to E/\tr{G}$. По кривой $E/\tr{G}$ и образам $\varphi (P), \varphi(Q)$
найти порождающий элемент  подгруппы $ G$ или, что эквивалентно, изогению $\phi:E\to E/G$).
 
 \subsection*{Тотальный перебор}
  
Поскольку у любой суперсингулярной кривой $E (GF(p^2))$ имеется $(l+1)l^{e-1}$ циклических подгрупп порядка $l^e$, то тотальный перебор требует $O(l^e) $ или $O(p^{1/2})$ опробований.
 
 \subsection*{Метод ``встречи посередине''}

 Пусть для простоты $e$ четное.
 
Построим два дерева таких, что листья первого определяют классы изоморфизмов кривых, $l^{e/2}$-изогенных $E$;
листья второго -- классы изоморфизмов кривых, $l^{e/2}$-изогенных $E/G$.

В каждом наборе по $(l+1)l^{e/2-1}$ классов; с большой вероятностью единственный класс, заданный представителями $E'$ и $E''$ из первого и второго наборов соответственно, содержится в их пересечении. Найдя его, строим 
$\phi$ как композицию изогении из $\phi_1:E\to E'$, изоморфизма $\psi:E'\to E''$ и двойственной к изогении из $E/G$: ${\phi}_2:E/\tr{G}\to E''$.

Требуемый объем памяти --- $O(p^{1/4})$ ячеек, время --- $O(p^{1/4})$ операций.

%В \cite{Adj} предложен метод построения 

 \subsection*{Квантовый вычислитель}

\emph{Claw-finding} алгоритм из \cite{quant} для заданных функций $g_1:X_1\to Y$, $g_2:X_2\to Y$ определяет такие $(x_1, x_2)\in X_1\times X_2$: $g_1(x_1)=g_2(x_2)$.

Пусть при этом $\#X_1\approx \#X_2 \approx N$, $\#Y\gg N$, тогда время работы составляет $O(N^{2/3})$ операций при требованиях к памяти $O(N^{2/3})$.

В нашем случае $X_1 $ -- множество $l^{e/2}$-изогений  из $E=E_1$; $X_2$ -- множество $l^{e/2}$-изогений  из $E/G=E_2$, $g_i(\phi)=j(\phi(E_i))$.
Имеем $\#X_1=\#X_2\approx p^{1/4}$, отсюда  время -- $O(p^{1/6})$ (и память $O(p^{1/6})$).

Метод Гровера, примененный к задаче CSSI в \cite{FFF}, требует $O(p^{1/4})$ операций,  необходимая память -- $O(1)$.

В работе \cite{FFF} проведен сравнительный анализ квантовых  алгоритмов, из которого следует, что в реалистичной модели квантового вычислителя (с ограниченным объемом памяти, не превосходящим
$2^{64}$ ячеек) метод Гровера является более эффективным.

 \subsection*{Параллельный поиск коллизий}
 
Параллельный метод поиска коллизий (метод Винера--ван Ооршота)  из \cite{vOW}, эффективно применяется к задаче CSSI в \cite{Adj,Costello}, коэффициент
распараллеливания при этом близок к 100\%.

Наиболее эффективные методы поиска коллизий псевдослучайной функции $f$ являются итерационными в том смысле, что они основаны на вычислении 
последовательностей вида $x_i = f(x_{i-1})$, $i \in \NN$, откуда следует, что функция $f$ должна быть такой, что множество ее значений содержится в множестве, 
на котором она определена.
 
Идея метода состоит в том, что
каждый процессор вырабатывает свою последовательность $x_i=f(x_{i-1})$ до появления \emph{выделенной точки} $x_d$, удовлетворяющей некоторому легко проверяемому условию (например, фиксированное число нулевых старших битов), $x_d$ записывается в общую для всех процессоров память по адресу, вычисляемому как некоторая взаимно однозначная функция точки, и начинает выработку новой последовательности.


Если одна и та же выделенная точка встречается дважды -- найдена коллизия $f$. Каждый процессор вместе с последовательностью $x_i$ вычисляет значения $(x^{(i)},y^{(i)})$ и сохраняет их для каждой выделенной точки, совпадение выделенных точек $x^d$ и $x'_{d'}$, означает, что $((x^{(d)}, y^{(d)}), (x^{(d')}, y{(d')}))$ -- коллизия.

Среднее время работы алгоритма приближенно равно
$$
\left(\sqrt{\frac{\pi\#S}{2p}}/m+\frac{\alpha}{\theta}\right)t,
$$
где
\begin{itemize}
\item $f$ -- случайное отображение,
\item $p$ -- вероятность того, что случайно выбранная коллизия $f$ является \emph{полезной},
\item $\theta$ -- доля выделенных точек в множестве $S$,
\item $t$ -- время одной итерации.
\end{itemize}

Пусть $S=\{0,1\}\times\{0,\dots,(l+1)l^{e/2-1}-1\}$, $E_0=E$, $E_1=E/G$.  Каждая пара $(i,y)\in S$ задает подгруппу эллиптической кривой $E_i$. 
\begin{example}
 Для $l=2$ в \cite{Adj} соответствие задается между парами $(i,y)=(i,(b,k))\in\{0,1\}\times\{0,1,2\}\times \{0,\dots,l^{e/2-1}-1\}$ и 
 циклическими подгруппами $\tr{R_i}\subset E_i$, где
 \begin{equation*}
R_i = 
 \begin{cases}
   P_i+[b2^{e/2-1}k,  &\text{если } b=0,1 \\
   [2k]P_i+Q_i,  &\text{если } b=2,
 \end{cases}
\end{equation*}
где $\tr{P_i,Q_i} = E_i[2^{2/2-1}]$.
\end{example}

Пусть $h:S\to E_0(GF(p^2))\cup E_1(GF(p^2))$, $h:(i,y)\mapsto R_i$, и  пусть итерационная функция $f:S\to S$ -- функция, которая по входной паре $(i,y)$
вычисляет изогению степени $l^{e/2}$ с ядром $\tr{R_i}$, вычисляет $j$-инвариант $j(E_i/\tr{R_i})$ и отображает его в $S$ при помощи некоторой псевдослучайной
функции $g: GF(p^2)\to S$.

%Коллизия для $f$ -- это пара $x,x'$ такая, что 
Существует единственная полезная коллизия для $f$, которая и решает задачу CSSI.

Таким образом, итоговые оценки трудоемкости параллельного метода поиска коллизий применительно к задаче CSSI:
\begin{equation}\label{eq1}
T=\frac{2.5}{m}\sqrt{|S|^3/w}\cdot t;
\end{equation}
$m$ -- количество процессоров, $|S|$ -- мощность множества определения итерационной функции, $w$ -- объем памяти, $t$ -- трудоемкость итерационной функции.

В нашем случае $|S|\approx p^{1/2}$, и таким образом оценка стойкости составляет
\begin{equation}\label{eq2} 
O\left(\frac{p^{3/8}}{m\ w^{1/2}}\right)
\end{equation}
операций вычисления итерационной функции (в нашем случае -- вычисления $l$-изогении).

\subsection*{Несбалансированные делители и слабые стартовые кривые}


В работе \cite{Kutas} предлагается атака на имплементации SIDH со специально  подобранными параметрами, трудоемкость которой меньше, чем у рассмотренных методов.
Для противодействия указанной атаке мы будем использовать хорошо сбалансированные простые $p$: $p=2^{e_2}3^{e_3}f-1$ так, что множители $2,3$ сбалансированы: 
$e_2\approx e_3 \cdot \log_2 3$, а множитель $f$ -- малое простое число,
и стартовую кривую специального вида из \cite{StartingCurve}.

%К ``слабым'' стартовым кривым относятся 


\subsection*{Долговременные ключи}

Отметим, что эффективная атака, предложенная в \cite{sec}, позволяет определить долговременный ключ абонента за $O(\log p)$ сеансов, поэтому использование долговременных ключей,
равно как и повторное использование эфемерных ключей, протоколом не допускается.

\subsection*{Выводы}

Следуя рассуждениям работ \cite{Adj, FFF}, можно заключить, что наилучший классический метод решения задачи CSSI --- параллельный метод поиска коллизий, 
наилучший квантовый метод решения указанной задачи ---
метод Тани. Соответственно, при расчете параметров мы будем ориентироваться на трудоемкость указанных методов.


\section{Выбор параметров}\label{parampampam}

Простое число  $p$ выбирается в виде $p=2^{e_2}3^{e_3}f-1$ так, что множители $2,3$ сбалансированы:  $e_2\approx e_3 \cdot \log_2 3$, а множитель $f$ -- малое простое число. 
При этом для $p$ требуется, чтобы
$\big(\frac{-19}{p}\big) = -1$ (см. далее).

В соответствии с оценками раздела \ref{analysis}, предлагаются следующие наборы параметров. %, обеспечивающих классический уровень стойкости, соответственно, 128 и 256 битов:

\begin{center}
 \begin{tabular}{|l|c|p{3cm}|p{3cm}|}
 \hline
 Число & Формула & Классическая стойкость & Квантовая стойкость \\
 \hline
 $p_{485}$ & $2^{242} \cdot 3^{152} \cdot 5 -1$ & 173  & 80 \\
 \hline
 $p_{772}$ & $2^{386} \cdot 3^{242} \cdot 7-1$ & 281 & 128 \\
 \hline
 $p_{1532}$ & $2^{776} \cdot 3^{496} -1 $ & 566 & 255 \\
 \hline
 %
 %$p_{}$ & & & \\
 %\hline
 %$p_{415}$ & $2^{208}\cdot 3^{129} \cdot 5 - 1$ & 128 & 103 \\
 %\hline 
 %$p_{754}$ & $2^{372} \cdot 3^{239} \cdot 7 - 1$ & 256 & 187\\
 %\hline
 \end{tabular}

\end{center}

\begin{remark}
 Классическая стойкость рассчитана по формуле (\ref{eq1}) для параметров суперкомпьютера Fugaku, возглавляющего текущую редакцию списка Top500 \cite{Top500} с $\approx 2^{23}$ вычислительными ядрами
 и $\approx 2^{49}$ ячеек памяти при $t=2^{22}$.
\end{remark}


% 
% $$
%  p_{415}=2^{208}\cdot 3^{129} \cdot 5 - 1
% $$
% $$
% p_{754}= 2^{372} \cdot 3^{239} \cdot 7 - 1
% $$


Стартовая кривая протокола может фиксироваться, например, как $E_0(GF(p^2)): y^2=x^3+x$. %; она суперсингулярна, поскольку $p_\lambda\equiv 3\pmod{4}$.%, см. \cite{Silverman}.
Однако мы предлагаем использовать кривую   $E_{19}(GF(p))$:
\begin{equation}
E_{19}\!: y^2 = x^3 - 2^3 19 x + 2 \!\cdot\! 19^2
\end{equation}
с $j$-инвариантом $-2^{15} 3^3$. Ее выбор мотивирован в \cite{StartingCurve}. В частности,  $E_{19}$ не обладает эндоморфизмами степени 2 и 3, то есть петлями в
графах 2- и 3-изогений.  Также у  нее отсутствуют кратные дуги в графах 2- и 3-изогений, в отличие от   кривых, использованных в SIKE \cite{SIKE}.
 
По теореме Дойринга (см., например, \cite[Теорема 2.1]{broker}) для суперсингулярности кривой $E_{19}$ необходимо и достаточно, чтобы $\big(\frac{-19}{p}\big) = -1$. %В свою очередь, благодаря квадратичному закону взаимности это эквивалентно тому, что $\big(\frac{p}{19}\big) = -1$.
%Указанное условие выполняется для $p_{415}, p_{754}$.

Эллиптические кривые задаются   в форме Монтгомери (\ref{monty}).

Для построения базиса $\{P_A,Q_A\}$ будем действовать следующим образом.
Будем перебирать случайные точки   $P\in_R E_0$ и вычислять $P'=[(l_B ^{e_B}\cdot f)^2]P$; с большой вероятностью это точка порядка $((l_A)^{e_A})^2$ (проверяем, умножая на степени $l_A$), и тогда
 $P_A=P'$.
 Аналогично вычислим $Q_A$ порядка $((l_B)^{e_B})^2$. 
 
 Проверка независимости: \emph{спаривание Вейля} $e(P_A,Q_A)$ в $E[l_A^{e_A}]$ должно иметь порядок $l_A^{e_A}$ (с большой вероятностью это так).
 
Аналогично выберем базис $\{P_B,Q_B\}$.

\begin{remark}
 Предполагается заменить данную процедуру на детерминированную, аналогично тому, как сделано в SIKE \cite{SIKE}.
\end{remark}



%% FIXME!


\section{Реализация}

TODO!!!

\section*{Заключение}\addcontentsline{toc}{section}{Заключение}

Основные результаты работы подгруппы на данном этапе:

\begin{itemize}
 \item  предложен постквантовый протокол выработки общего ключа двумя абонентами на основе протокола SIDH;

 \item  выбраны параметры протокола, в том числе, стартовая кривая, характеристики поля и т.д.

 \item  разработан прототип программной реализации на языке Python с использованием библиотеки SAGE.
\end{itemize}


Основные задачи для дальнейшей работы:
\begin{itemize}

 \item  исследование криптографической стойкости протокола;

 \item  разработка оптимизированной программной реализации протокола.
\end{itemize}

\newpage

  \renewcommand{\baselinestretch}{.9}\large
\newcommand{\noopsort}[1]{} \newcommand{\printfirst}[2]{#1}
  \newcommand{\singleletter}[1]{#1} \newcommand{\switchargs}[2]{#2#1}
\begin{thebibliography}{1}\addcontentsline{toc}{section}{Список литературы}


%\bibitem{Gost2012}
%ГОСТ Р 34.10-2012. Информационная технология.
%  Криптографическая защита информации.
%  Процессы формирования и проверки
%  электронной цифровой подписи.%
	%\newblock  --- Стандартинформ, Москва.
%\newblock --- 2013.

%\bibitem{GostHash2012}
%ГОСТ Р 34.11-2012. Информационная технология.
%  Криптографическая защита информации.
%  Функция хэширования.%
%\newblock  --- Стандартинформ, Москва.
%\newblock --- 2013.

%\bibitem{GostCrypto}
%ГОСТ Р 34.12-2015. Системы обработки информации.
%  Защита криптографическая. Блочные шифры.
%\newblock  --- Стандартинформ, Москва.
%\newblock --- 2015.

%\bibitem{GostCryptoRez} ГОСТ Р 34.13-2015.
%  Информационная технология.
%  Криптографическая защита информации.
%  Режимы работы блочных шифров.
%\newblock  --- Стандартинформ, Москва.
%\newblock --- 2015.

%\bibitem{NIST} Submission Requirements and Evaluation Criteria for the Post-Quantum Cryptography Standardization Process. --- NIST. --- 2017.

\bibitem{EFD} Explicit-Formulas Database. XZ coordinates for Montgomery curves. \url{http://hyperelliptic.org/EFD/g1p/auto-montgom-xz.html}. --- 2020.

\bibitem{Top500} Supercomputer Fugaku. \url{https://www.top500.org/system/179807/}. --- 2020.

%\bibitem{bok}R. broker. Constructing supersingular elliptic curves. --- J. Comb. Number Theory, 1(3):269-273. --- 2009.



\bibitem{Adj} Adj~G., Cervantes-Vázquez~D., Chi-Domínguez~J-J.,  Menezes~A., Rodríguez-Henríquez~F. On the cost of computing isogenies between supersingular elliptic curves.  \url{http://eprint.iacr.org/2018/313}. --- 2018.


\bibitem{broker} Bröker~R. {\it Constructing supersingular elliptic curves}. // Journal of Combinatorics and Number Theory, 2009. Vol. 1(3). P. 269--273.

\bibitem{CK}
Canetti~R., Krawczyk~H.
\newblock
\newblock Analysis of key-exchange protocols and their use for building secure
 channels// EUROCRYPT 2001, LNCS 2045.
\newblock --- N. Y.: Springer-Verlag.
\newblock --- 2001.
\newblock ---  P.~453--474.


\bibitem{Costello0} Costello~C., Longa~P., Naehrig~M. Efficient algorithms for supersingular isogeny Diffie-Hellman. \url{http://eprint.iacr.org/2016/413} (2016).


\bibitem{Costello} Costello~C., Longa~P., Naehrig~M., Renes~J., Virdia~F. Improved Classical Cryptanalysis of SIKE in Practice. \url{http://eprint.iacr.org/2019/298} (2019).

%Costello C. et al. Improved Classical Cryptanalysis of SIKE in Practice //IACR International Conference on Public-Key Cryptography. – Springer, 2020. – P. 505-534.

%\bibitem{Imp2}
% Costello C., Longa P., Naehrig M. Efficient algorithms for supersingular isogeny Diffie-Hellman. --- 
% Cryptology ePrint Archive: Report 2016/413. 
% \newblock \url{https://eprint.iacr.org/2016/413}. --- 2016.


%\bibitem{Costello2} Costello~C. Supersingular isogeny key exchange for beginners. \url{http://eprint.iacr.org/2019/1321}. --- 2019.

%\bibitem{Costello3} Costello~C., Smith~B. Montgomery curves and their arithmetic: The case of large characteristic fields. \url{http://eprint.iacr.org/2017/212} --- 2017.
 
 
  \bibitem{DF} De Feo~L., Jao~D., Plût~J.   Towards quantum-resistant cryptosystems from supersingular elliptic curve isogenies.  // Journal of Math. Cryptology, 2014. Vol. 8(3). P. 209--247. 

% \bibitem{Denchik}
% \newblock Denisenko~D., Marshalko~G., Nikitenkova~M., Rudskoy~V., Shishkin~V.
% \newblock Estimation of Grover's algorithm implementation for searching GOST R 34.10-2015 block cipher keys
% \newblock --- Journal of Experimental and Theoretical Physics. --- 2019. --- Vol. 128, no. 4. --- P. 552–559.
%  


% \bibitem{Imp5} ``defeo/ss-isogeny-software''. GitHub. \url{https://github.com/defeo/ss-isogeny-software}. --- 2015.

 %\bibitem{Gal1}
%Delfs C., Galbraith S. D. Computing isogenies between supersingular elliptic curves over $\mathbb{F}_p$ //Designs, Codes and Cryptography. – 2016. – Т. 78. – №. 2. – С. 425-440. 

% \bibitem{Edwards2007}
%Edwards H. M. \newblock A normal form for elliptic curves.
% --- bulletin of the AMS, 44: 393–422. --- 2007.
  
%\bibitem{Imp1}
% Fishbein  D. 
% \newblock Machine-Level Software Optimization of Cryptographic Protocols. University of Waterloo Library - Electronic Theses.
% University of Waterloo. --- 2014.

% \bibitem{Gal2}
% Galbraith S., Petit C.,  Silva J.
% \newblock Schemes based On Supersingular Isogeny Problems, Cryptology ePrint Archive: Report 2016/1154. 
% \newblock \url{https://eprint.iacr.org/2016/1154}. --- 2016.

\bibitem{sec}
Galbraith~S., Petit~C.,  Shani~B., Yan Bo Ti.
\newblock
On the Security of Supersingular Isogeny Cryptosystems. --- 
Cryptology ePrint Archive: Report 2016/859. --- \newblock \url{https://eprint.iacr.org/2016/859}. --- 2016.

  
%  \bibitem{Gal18}
% Galbraith S.
%\newblock Authenticated key exchange for SIDH. ---
% Cryptology ePrint Archive: Report 2018/266
%\newblock \url{http://eprint.iacr.org/2018/266}.
%--- 2018.

\bibitem{SIKE}  Jao~D. et al. \newblock Supersingular Isogeny Key Encapsulation\newblock \url{https://sike.org} --- 2017.


\bibitem{FFF}
Jaques~S.,  Schanck~J.M.
\newblock Quantum cryptanalysis in the RAM model: Claw-finding attacks on SIKE.//
 Cryptology ePrint Archive: Report 2019/103. ---  \newblock \url{https://eprint.iacr.org/2019/103}. --- 2019.

 
% \bibitem{JKL} 
% Ik Rae Jeong, Katz J., Dong Hoon Lee. \newblock One-round protocol for two-party key exchange. // In ACNS 2004, volume 3089 of LNCS.
% \newblock 2004.
% \newblock P.~220-232.
 
%\bibitem{Static}
%Kirkwood D., Lackey b.C., McVey J., Motley M., Solinas J.A., Tuller D.
%\newblock Failure is not an option: Standardization issues for post-quantum key agreement//
%Workshop on Cybersecurity in a Post-Quantum World. --- 2015.

 %\bibitem{Imp3} Koziel b., Jalali A., Azarderakhsh R., Kermani M., Jao D.  \newblock EON-SIDH: Efficient Implementation of Supersingular Isogeny Diffie-Hellman Key Exchange Protocol on ARM. --- 
 %Cryptology ePrint Archive: Report 2016/413. --- 
 %\url{https://eprint.iacr.org/2016/413} --- 2016
 
 %\bibitem{Imp4} Koziel b., Jalali A., Azarderakhsh R., Kermani M. \newblock Fast Hardware Architectures for Supersingular Isogeny Diffie-Hellman Key Exchange on FPGA. --- 
 %Cryptology ePrint Archive: Report 2016/1044. --- 
 %\url{http://eprint.iacr.org/2016/1044} --- 2016.

\bibitem{Kutas} Kutas P., Martindale C., Panny L., Petit C.,  Stange K.E. Weak instances of SIDH variants under improved torsion-point attacks. \url{http://eprint.iacr.org/2020/633}. --- 2020.

 
%\bibitem{RoSt}  Rostovtsev A., Stolbunov A.: Public-key cryptosystem based on isogenies. --- Cryptology ePrint Archive, Report 2006/145. --- \url{http://eprint.iacr.org/2006/145}. --- 2006.

\bibitem{Miller} Miller V.  The Weil Pairing, and Its Efficient Calculation. J. Cryptology. 17. 235-261. --- 2004. 

%%\bibitem{Mon0} Montgomery P.~L.  Modular multiplication without trial division. Mathematics of computation, 44(170):519–521, 1985.

\bibitem{Mon}
Montgomery P.~L.
\newblock
\newblock Speeding the {P}ollard and elliptic curve methods of factorization//
  Math. Comp.
\newblock --- 1987. --- {\bf 48}. --- pp.~243--264.


\bibitem{vOW} van Oorschot~P., Wiener~M. Parallel Collision Search with Cryptanalytic Applications. J. Cryptology 12, 1–28 (1999).

\bibitem{Piz1} Pizer A. Ramanujan graphs and Hecke operators. Bull. Amer. Math. Soc. (N.S.), 23(1):127–137, 1990.

\bibitem{Piz2}  Pizer A. Ramanujan graphs. In Computational perspectives on number theory (Chicago, IL, 1995), volume 7 of
AMS/IP Stud. Adv. Math., pages 159–178. Amer. Math. Soc., Providence, RI, 1998.

%\bibitem{Renes} Renes~J. Computing isogenies between Montgomery curves using the action of (0,0). \url{http://eprint.iacr.org/2017/1198}. ---   2017.


\bibitem{Silverman}
Silverman~J.H.
\newblock The Arithmetic of Elliptic Curves. 
--- Springer:2009.


%  \bibitem{DSIG1} Sun Xi, Tian Wang. \newblock
%  Toward Quantum-Resistant Strong Designated Verifier Signature from Isogenies. Intelligent Networking and Collaborative Systems (INCoS), 2012 4th International Conference on. IEEE: 292–296. 

\bibitem{quant}
  Seiichiro Tani. 
  \newblock Claw Finding Algorithms Using Quantum Walk. \url{http://arxiv.org/abs/0708.2584}. --- 2008.
  
  
\bibitem{Velu}   Vel\'{u}~J. \newblock Isog\'{e}nies entre courbes elliptiques, C.R. Acad. Sc. Paris, Serie A., 273, pp. 238–241 (1971).

\bibitem{Wash}
Washington~L.C.
\newblock Elliptic curves, number theory and cryptography: CRC.
\newblock --- 2008.

% \bibitem{DSIG2} Youngho Yoo, Azarderakhsh R.,  Jalali A., Jao D., Soukharev V. A Post-Quantum Digital Signature Scheme
%based on Supersingular Isogenies. 
% Cryptology ePrint Archive: Report 2017/186. --- 
%\url{https://eprint.iacr.org/2017/186}. --- 2017.

\bibitem{StartingCurve} Кошелев~Д. {\it Стартовая суперсингулярная эллиптическая кривая для криптографии на изогениях}, \url{https://www.researchgate.net/profile/Dimitri_Koshelev}, 2020.

%%%%%%%%%%%%%%%%%%%%%%%%%%%
\end{thebibliography}

\newpage 

\appendix
\section{Контрольные примеры}

TODO!!!

\end{document}

